\documentclass[12pt,a4paper]{article}

\usepackage{polski}
\usepackage[utf8]{inputenc}
\usepackage{indentfirst} 

\usepackage{fancyhdr}
\usepackage{graphicx} 
\usepackage[obeyspaces]{url}
\usepackage{lastpage}

\graphicspath{ {./resources/} }

\pagestyle{fancy}
\fancyhf[]{}
\linespread{1.5}
\setlength\headheight{14.5pt}
\cfoot{Strona \thepage \hspace{1pt} z \pageref{LastPage}}
\renewcommand\contentsname{}
\chead{Bieszczadzki Tour --- specyfikacja implementacyjna --- Maciej Czarkowski}

\begin{document}

\begin{titlepage}
\vspace*{\fill}
\begin{center}
{\fontsize{50}{0.1}\selectfont Bieszczadzki Tour}
\huge Specyfikacja implementacyjna
\end{center}
\vspace*{\fill}
\begin{center}
Maciej Czarkowski, 18.11.2019
\end{center}
\end{titlepage}
\clearpage
\hspace{1cm}
\begin{center}
\LARGE\textbf{Spis treści}
\end{center}
\tableofcontents
\clearpage
\section{Wstęp}
Niniejszy dokument, będący specyfikacją implementacyjną projektu \textsl{,,Bieszczadzki Tour''}, ma za zadanie możliwie najlepiej przybliżyć, osobom odpowiedzialnym za jego implementację, sposoby oraz metody prowadzące do stworzenia wydajnego i poprawnie działającego kodu. Program ma rozwiązywać problem odnalezienia optymalnej ścieżki pomiędzy zestawem zadanych punktów, w taki sposób, aby trasa była najkrótsza oraz możliwie najtańsza. Zgodnie z informacjami zawartymi w specyfikacji funkcjonalnej projektu, program do działania wykorzystuje pliki wejściowe, których konfiguracja powinna być zgodna z wyżej wymienionym dokumentem. Pożądanym efektem działania programu jest plik wynikowy, informujący użytkownika, którą trasą się udać, aby droga była optymalna.\\
\section{Środowisko deweloperskie}
Implementacja programu będzie odbywała się na komputerze \textsl{Dell Vostro 3578}, z 4-rdzeniowym procesorem \textsl{Intel Core i5-8250U}, korzystającym z systemu \textsl{Windows 10 Pro} w wersji 64-bitowej \textsl{10.0.18362}. Program zaimplementowany będzie w języku \textsl{Java} w wersji 8. Implementacja będzie odbywała się w środowisku programistycznym \textsl{IntelliJ IDEA 2018.3 (Community Edition)
Build \#IC-183.4284.148}, wydanym 21 listopada 2018 roku z wykorzystaniem narzędzi deweloperskich z pakietu \textsl{OpenJDK 64-Bit Server VM by JetBrains s.r.o Windows 10 10.0}. Środowiskiem uruchomieniowym dla kodu będzie maszyna wirtualna Javy w wersji \textsl{1.8.0\_152-release-1343-b15 amd64}.
\section{Zasady wersjonowania}
\section{Diagram klas}
\section{Istotne struktury danych w programie}
\section{Wykorzystane algorytmy}
\end{document}

