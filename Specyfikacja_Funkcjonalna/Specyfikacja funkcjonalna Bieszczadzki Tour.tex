\documentclass[12pt,a4paper]{article}

\usepackage{polski}
\usepackage[utf8]{inputenc}
\usepackage{indentfirst} 

\usepackage{fancyhdr}
\usepackage{graphicx} 
\usepackage[obeyspaces]{url}
\usepackage{lastpage}

\graphicspath{ {./resources/} }

\pagestyle{fancy}
\fancyhf[]{}
\linespread{1.5}
\setlength\headheight{14.5pt}
\cfoot{Strona \thepage \hspace{1pt} z \pageref{LastPage}}
\renewcommand\contentsname{}
\chead{Bieszczadzki Tour --- specyfikacja funkcjonalna --- Maciej Czarkowski}

\begin{document}

\begin{titlepage}
\vspace*{\fill}
\begin{center}
{\fontsize{50}{0.1}\selectfont Bieszczadzki Tour}
\huge Specyfikacja funkcjonalna
\end{center}
\vspace*{\fill}
\begin{center}
Maciej Czarkowski, 12.11.2019
\end{center}
\end{titlepage}
\clearpage
\hspace{1cm}
\begin{center}
\LARGE\textbf{Spis treści}
\end{center}
\tableofcontents
\clearpage
\section{Wstęp}
\subsection{Cel dokumentu}
Celem niniejszego dokumentu, będącego specyfikacją funkcjonalną projektu \textsl{,,Bieszczadzki Tour''}, realizowanego na potrzeby zadania projektowego z przedmiotu Algorytmy i Struktury Danych, jest wprowadzenie użytkownika końcowego w jego problematykę i możliwości. Ma on za zadanie również pełne poinstruowanie użytkownika, w jaki sposób odpowiednio i bezproblemowo korzystać z omawianego programu.
\subsection{Cel projektu}
Celem projektu jest pomoc zakłopotanym studentom w odnalezieniu optymalnej trasy, pomiędzy wybranymi przez nich punktami turystycznymi na~mapie Bieszczad. Stworzony program ma umożliwiać wygenerowanie trasy przez wszystkie wybrane przez studentów miejsca w możliwie najkrótszym czasie, z uwzględnieniem kosztów podróży.
\subsection{Użytkownik docelowy}
Użytkownikiem końcowym programu \textsl{,,Bieszczadzki Tour''} jest grupa studentów wybierających się na wyprawę po Bieszczadach, którzy z wykorzystaniem programu mają możliwość zoptymalizowania swojej wędrówki. Użytkownikiem końcowym jest również mgr inż. Paweł Zawadzki, prowadzący zajęcia z Algorytmów i Struktur Danych w semestrze 2019Z, który to na~podstawie działania programu dokona jego oceny.
\newpage
\section{Uruchomienie programu}
W celu uruchomienia programu, na lokalnym komputerze należy otworzyć wiersz poleceń, a następnie przejść do folderu z projektem, podając ścieżkę do niego w pokazany poniżej sposób, z wykorzystaniem komendy \textsl{cd}.\smallskip\\
\smallskip
\path{cd C:\path\to\project\directory}\\
\indent Po przejściu do powyższego folderu należy uruchomić program, podając jednocześnie argumenty do niego, zgodnie z poniższym schematem.\smallskip\\
{\scriptsize{\path{java -jar dist\bieszczadzkitour.jar path\to\data.txt ID_miejsca path\to\wishlist.txt}}\smallskip\\}
\indent W powyższej komendzie \path{path\to\data.txt} to ścieżka do pliku, w którym użytkownik powinien umieścić dane wejściowe dla programu (nazywanego dalej plikiem konfiguracyjnym), \path{ID_miejsca} to identyfikator miejsca rozpoczęcia podróży, zgodny z informacjami zawartymi w pliku konfiguracyjnym, a~opcjonalnie dodajemy \path{path\to\wishlist.txt}, czyli ścieżkę do~pliku, w~którym użytkownik określa miejsca, które chce odwiedzić (nazywanego dalej listą życzeń). Formatowanie danych wejściowych dla programu jest omówione szczegółowo w punkcie trzecim niniejszego dokumentu.
\section{Dane wejściowe}
Danymi wejściowymi dla programu są wspomniane w poprzednim punkcie argumenty, które podajemy przy uruchamianiu. Pliki, będące argumentami do~programu, powinny być plikami tekstowymi, które muszą posiadać określone formatowanie zawartości, aby program mógł działać w pełni poprawnie. W poniższych podsekcjach bardziej szczegółowo omówiono te elementy.\\
\newpage
\subsection{Plik konfiguracyjny}
W tym pliku tekstowym (należy wykorzystać rozszerzenie \path{.txt}) użytkownik powinien umieścić wszelkie informacje (unikatowy numer ID, nazwę oraz opis miejsca) na temat punktów, które rozpatrujemy podczas działania programu, oraz informacje na temat czasów (w obu kierunkach) i cen przejść pomiędzy poszczególnymi punktami. Plik powinien być zgodny z poniższym schematem.
\begin{figure}[h!]
\includegraphics[width = \linewidth]{data.jpg}
\caption{Poprawne formatowanie pliku z danymi wejściowymi}
\end{figure}
\subsection{ID\_miejsca}
Jest to ciąg znaków, zgodny z analogicznym polem w pliku konfiguracyjnym, który powinien jednoznacznie identyfikować i określać miejsce rozpoczęcia i jednocześnie zakończenia podróży.
\newpage
\subsection{Lista życzeń}
W tym pliku tekstowym (ponownie należy wykorzystać rozszerzenie \path{.txt}) użytkownik określa jednoznacznie, za pomocą pola \path{ID_miejsca}, miejsca, które chce odwiedzić. Formatowanie pliku powinno być zgodne z poniższym schematem.\\
\begin{figure}[h!]
\includegraphics[scale = 1.1]{wishlist.jpg}
\caption{Poprawne formatowanie pliku z listą życzeń}
\end{figure}
\newpage
\section{Dane wyjściowe}
W wyniku działania programu tworzony jest tekstowy plik wynikowy o~nazwie \path{result.txt}, w którym przedstawiona jest pełna, najkrótsza i możliwie najtańsza trasa z wyliczonym czasem oraz kosztem wędrówki. Plik tworzony będzie w folderze projektu, w podfolderze output, a jego formatowanie będzie zgodne z poniższym schematem.\\
\begin{figure}[h!]
\includegraphics[scale = 1]{result.jpg}
\caption{Formatowanie pliku wynikowego}
\end{figure}
\section{Scenariusz uruchomienia}
Po uruchomieniu programu zgodnie z punktem 2 niniejszej specyfikacji, w~przypadku poprawnego formatowania plików, program nie wypisze żadnych komunikatów o błędach, działanie programu zakończy się bez wyświetlania dodatkowych informacji w konsoli, a w podfolderze \path{.\output} w folderze projektowym ukaże się nowy plik wynikowy \path{result.txt}.
\section{Opis sytuacji wyjątkowych}
W przypadku podania nieprawidłowych argumentów przy uruchamianiu programu, w powłoce tekstowej, w której uruchamiany jest program, wyświetlony zostanie stosowny komunikat o niepoprawnym sformatowaniu argumentów z informacją, w którym argumencie wystąpił błąd. Dzięki tej informacji, użytkownik może poprawić wadliwy argument i ponownie przystąpić do uruchomienia programu. Komunikat będzie zgodny, w zależności od zaistniałej sytuacji wyjątkowej, z jednym z poniżej podanych przykładów.
\subsection{Błąd w pliku konfiguracyjnym}
Jeśli plik konfiguracyjny będzie zawierał niepoprawnie sformatowane dane, w jakikolwiek sposób niezgodne z podanym w punkcie 3.1 schematem, bądź jeśli dane nie będą poprawne pod względem matematycznym (np. podany zostanie ujemny czas lub opłata za przejście), w konsoli zostanie wypisany poniższy komunikat.\\
{\footnotesize{\path{Please review your configuration file, it seems to be an error in it!}}}
\subsection{Błąd w ID\_miejsca}
Jeśli pole \path{ID_miejsca}, podane jako argument, będzie sformatowane nieprawidłowo lub jeśli nie będzie miało swojego odpowiednika w pliku konfiguracyjnym, wyświetlony zostanie poniższy komunikat.\\
\path{You've passed ID_miejsca argument wrongly, please correct it.}
\subsection{Błąd w Liście Życzeń}
W przypadku gdy plik \path{wishlist.txt} będzie sformatowany niezgodnie z~wymaganym schematem, bądź jeśli dane w nim zawarte nie będą zgodne z~plikiem konfiguracyjnym (miejsce z listy życzeń nie istnieje na liście wszystkich naszych miejsc), konsola wyświetli poniższy komunikat.\\
\path{Please review your wishlist, it seems to be an error in it!}
\subsection{Brak pliku konfiguracyjnego}
W przypadku braku argumentu będącego ścieżką do pliku konfiguracyjnego, zostanie wyświetlony poniższy komunikat.\\
\path{Configuration file is missing!}
\subsection{Brak argumentu ID\_miejsca}
Kiedy nie zostanie podane \path{ID_miejsca}, ukazany zostanie poniższy komunikat.\\
\path{ID_miejsca argument is missing!}
\section{Testowanie}
Program będzie testowany poprzez podawanie jako argumenty różnych zestawów danych --- zarówno poprawnych, jak i niepoprawnych. W przypadku podania danych o niepoprawnym formatowaniu sprawdzana będzie funkcjonalność, w jaki sposób program informuje nas o błędach w argumentach --- czy komunikat o błędzie jest wyświetlony prawidłowo oraz czy jednoznacznie informuje nas jaki błąd wystąpił.\\
\indent W przypadku testowania poprawnego działania programu przygotowane zostaną zestawy danych wejściowych o poprawnym formatowaniu oraz o~wiadomym poprawnym wyniku działania programu. Po wykonaniu programu, dane z pliku wynikowego zostaną porównane z wynikiem oczekiwanym, co pozwoli nam stwierdzić czy program zadziałał prawidłowo.
\end{document}

